\documentclass[../main.tex]{subfiles}
\begin{document}
    \begin{enunciado}
        Una función $\F f X \Rext$ es medible si y solo si, los conjuntos
            \begin{align*}
                A := \llave{x \in X : \f f x = + \infty} \text{ y } B := \llave{x \in X : \f f x = -\infty}
            \end{align*}
        pertenecen a $\A$ y la función con valores reales dada por
            \begin{align*}
                \f {\tilde f} x = \begin{funcionporpartes}{ll}
                                    \f f x & \si x \not\in A \uu B,\\
                                    0 & \si x \in A \uu B,
                               \end{funcionporpartes}
            \end{align*}
        es medible.
    \end{enunciado}
    
    \begin{demostracion}
        Sea $f \in \E[\M] X \A$. Observe que
            \begin{align*}
                A &= \llave{x \in X : \f f x = + \infty} = \Ii n 1 \infty \llave{x \in X : \f f x > n},\\
                B &= \llave{x \in X : \f f x = - \infty} = \cuadro{ \Ui n 1 \infty \llave{x \in X: \f f x > -n} }^c
            \end{align*}
        de esta forma, $A,B \in \A$. Ahora, se probará que $\tilde f$ es medible. Para $\alpha \in \R$, se tendrá que
            \begin{align*}
                \llave{x \in X : \f {\tilde f} x > \alpha} = \begin{funcionporpartes}{ll}
                                                            \llave{x : \f f x > \alpha} \setminus A & \si \alpha \geq 0,\\
                                                            \llave{x : \f f x > \alpha} \uu B & \si \alpha < 0.
                                                          \end{funcionporpartes}
            \end{align*}
        Así, concluimos que es medible.
    \end{demostracion}

\end{document}
