\documentclass[../template.tex]{subfiles}
\begin{document}
    \begin{enunciado}
    Una funcion $\funcion{f}{X}{\ol{\real}}$ es medible si y solo si, los conjuntos
        \begin{align}
            A := \llave{x \in X : f(x) = + \infty} \text{ y } B := \llave{x \in X : f(x) = - \infty}
        \end{align}
    pertenecen a $\mathcal{A}$ y la función con valores reales dada por
        \begin{align}
            \tilde{f}(x) = \begin{funcionporpartes}{ll}
                                f(x) & \text{ si } x \not\in A \union B,\\
                                0 & \text{ si } x \in A \union B,
                           \end{funcionporpartes}
        \end{align}
    es medible.
    \end{enunciado}

    \begin{demostracion}
        Se tiene
            \begin{align}
                a = 0
            \end{align}
        de esta forma
            \begin{align}
                b = 0
            \end{align}

    \end{demostracion}

\end{document}
