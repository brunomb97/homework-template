\documentclass[../template.tex]{subfiles}
\begin{document}
    \begin{enunciado}
        Una función $\funcion{f}{X}{\ol{\rr}}$ es medible si y solo si, los conjuntos
            \begin{align*}
                A := \llave{x \in X : f(x) = + \infty} \text{ y } B := \llave{x \in X : f(x) = -\infty}
            \end{align*}
        pertenecen a $\algebra$ y la función con valores reales dada por
            \begin{align*}
                \tilde{f}(x) = \begin{funcionporpartes}{ll}
                                    f(x) & \si x \not\in A \union B,\\
                                    0 & \si x \in A \union B,
                               \end{funcionporpartes}
            \end{align*}
        es medible.
    \end{enunciado}
    
    \begin{demostracion}
        \demder
        Sea $f \in \medible{X}{\algebra}$. Observe que
            \begin{align*}
                A &= \llave{x \in X : f(x) = + \infty} = \biginter_{n=1}^{\infty} \llave{x \in X : f(x) > n},\\
                B &= \llave{x \in X : f(x) = - \infty} = \cuadro{ \bigunion_{n=1}^{\infty} \llave{x \in X: f(x) > -n} }^c
            \end{align*}
        de esta forma, $A,B \in \algebra$. Ahora, se probará que $\tilde{f}$ es medible. Para $\alpha \in \rr$, se tendrá
            \begin{align*}
                \llave{x \in X : \tilde{f}(x) > \alpha} = \begin{funcionporpartes}{ll}
                                                            \llave{x : f(x) > \alpha} \backslash A & \si \alpha \geq 0,\\
                                                            \llave{x : f(x) > \alpha} \union B & \si \alpha < 0.
                                                          \end{funcionporpartes}
            \end{align*}
        Así, concluimos que $\tilde{f}$ es medible.
        
        \demizq
        
    \end{demostracion}

\end{document}
