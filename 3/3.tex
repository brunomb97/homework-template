\documentclass[../main.tex]{subfiles}
\begin{document}
    Una funcion $\F f X {\ol{\R}}$ es medible si y solo si, los conjuntos
        \begin{align}
            A := \llave{x \in X : f(x) = + \infty} \text{ y } B := \llave{x \in X : f(x) = - \infty}
        \end{align}
    pertenecen a $\A$ y la función con valores reales dada por
        \begin{align}
            \tilde{f}(x) = \left\{ 
                \begin{array}{ll} 
                    f(x) & \text{ si } x \not\in A \union B \\ 
                    0 & \text{ si } x \in A \union B 
                \end{array} \right.
        \end{align}
    es medible.
    
    \begin{demostracion}
        Se tiene
            \begin{align}
                a = 0
            \end{align}
        de esta forma
            \begin{align}
                b = 0
            \end{align}

    \end{demostracion}

\end{document}
